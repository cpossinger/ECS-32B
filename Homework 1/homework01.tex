
\documentclass[11pt]{article}

\usepackage[utf8]{inputenc}
\usepackage[T1]{fontenc}
\usepackage[english]{babel}


% -- Packages
\usepackage{amssymb}
\usepackage{mathpazo}
\usepackage{listings}
\usepackage{enumitem}

% Commands
\newcommand{\ans}{\textbf{Answer}}



\title{ECS 32B -- Introduction to Data Structures \\ Homework 01}
\author{Name:  Camden Possinger\\
Student ID: 916955731 }
\date{Due: October 9, 2020, 5:00pm PST}

\usepackage[margin=2cm]{geometry}

\begin{document}
\maketitle

\textit{Make sure to clearly identify your name and student number. Submit only one
pdf file via Gradescope. Do not submit a separate file for each problem. If you
handwrite your solutions, make sure they are clear and readable.}

%--------------------------------------------------
% Problem 1
%--------------------------------------------------
\section*{Problem 1 (10 points each)}
For each of the following functions, do the following:
      \begin{itemize}
      \item Calculate $T(n)$, making sure not to discard constants or low-order terms
            yet (you may or may not count the counting variable in a for loop)
      \item Given what you think is the Big-O performance for the function
      \item Prove that your proposed Big-O for the function is correct (make sure to
            show your work; do not just provide values for $c$ and $n_0$)
      \end{itemize}
      \vspace{20pt}
      
      \begin{enumerate}[label=(\arabic*)]
      \item \begin{lstlisting}[language=Python]
            def fun1(n):
                x = 0  +1
                for i in range(n/2):  +(n/2)
                    x = 2 * 5 - i +1
                return x +1
            \end{lstlisting}

 
\begin{itemize}
\item $T(n)$ is $\frac{n}{2} + 2$
\item $O(n)$ is $n$
\item Prove $T(n)$ is $O(n)$ showing that $T(n) \leq c \cdot f(n)$ where $c > 0$ and $n_{0} \geq 0$ such that $ \forall  n \geq n_{0}$ \\ 
 
\centering $\frac{n}{2} +2 \leq c \cdot n$  \\
\vspace{5pt}
when $n_{0} = 2$ and $c = 2$ \\
\vspace{5pt}
\centering $ 3 \leq 4$ is true\\
\vspace{5pt}
 $c > 0$ and $ n_{0} \geq 0 $ where $  n \geq n_{0}$  \\
\vspace{5pt}
 $T(n)$ is $O(n)$  $\blacksquare$
							
\end{itemize}

      \item \begin{lstlisting}[language=Python]
            def fun2(n):
                x = 0 +1
                for i in range(1, n): +(n-1)
                    for j in range(1, n):  +(n-1)
                        for k in range(1, n):  +(n-1)
                            x = x + 1  +1
                return x   +1
            \end{lstlisting}

\begin{itemize}
\item $T(n)$ is $(n-1)^3 +2$ which simplifies to $n^3-3n^2+3n+1$
\item $O(n)$ is $n^3$

\item Prove $T(n)$ is $O(n)$ showing that $T(n) \leq c \cdot f(n)$ where $c > 0$ and $n_{0} \geq 0$ such that $ \forall  n \geq n_{0}$ \\ 

\centering $n^3-3n^2+3n+1\leq c \cdot n^3$  \\
\vspace{5pt}
when $n_{0} = 2$ and $c = 1$ \\
\vspace{5pt}
\centering $ 3 \leq 8$ is true\\
\vspace{5pt} 
since $c > 0$ and $ n_{0} \geq 0 $  where  $   n \geq n_{0}$ \\
\vspace{5pt}
 $T(n)$ is $O(n)$  $\blacksquare$ 
							
\end{itemize}

\noindent\makebox[\linewidth]{\rule{\paperwidth}{0.4pt}}

      \item \begin{lstlisting}[language=Python]
            def fun3(n):
                x = 0 +1
                i = n  +1
                while i >= 0: +(n+1) 
                    x = x - i +1
                    i = i - 1 +1
                return x +1
            \end{lstlisting}

\begin{itemize}
\item $T(n)$ is $2(n+1)+3$ which simplifies to $2n+5$
\item $O(n)$ is $n$

\item Prove $T(n)$ is $O(n)$ showing that $T(n) \leq c \cdot f(n)$ where $c > 0$ and $n_{0} \geq 0$ such that $ \forall  n \geq n_{0}$ \\ 

\centering $2n+5\leq c \cdot n$  \\
\vspace{5pt}
when $n_{0} = 3$ and $c = 4$ \\
\vspace{5pt}
\centering $ 11 \leq 12$ is true\\
\vspace{5pt} 
since $c > 0$ and $ n_{0} \geq 0 $ where $n \geq n_{0}$\\
\vspace{5pt}
 $T(n)$ is $O(n)$  $\blacksquare$

							
\end{itemize}

\noindent\makebox[\linewidth]{\rule{\paperwidth}{0.4pt}}

      \item \begin{lstlisting}[language=Python]
            def fun4(n):
                x = n +1
                for i in range(n): +n
                    x = x + 1 +1
                for i in range(n * n): +n^2
                    x = x + 2 +1
                return x +1
            \end{lstlisting}

\begin{itemize}
\item $T(n)$ is $n^2+n+2$
\item $O(n)$ is $n^2$

\item Prove $T(n)$ is $O(n)$ showing that $T(n) \leq c \cdot f(n)$ where $c > 0$ and $n_{0} \geq 0$ such that $ \forall  n \geq n_{0}$ \\ 

\centering $n^2+n+2\leq c \cdot n^2$  \\
\vspace{5pt}
when $n_{0} = 2$ and $c = 3$ \\
\vspace{5pt}
\centering $ 8 \leq 12$ is true\\
\vspace{5pt} 
since $c > 0$ and $ n_{0} \geq 0 $ where $ n \geq n_{0}$\\
\vspace{5pt}
 $T(n)$ is $O(n)$  $\blacksquare$

\noindent\makebox[\linewidth]{\rule{\paperwidth}{0.4pt}}
							
\end{itemize}

      \item \begin{lstlisting}[language=Python]
            # lst is a list of integers
            def fun5(lst):
                x = 0  +1
                for y in lst: +n
                    x = x + y +1
                return x +1
            \end{lstlisting}
\begin{itemize}
\item $T(n)$ is $n+2$
\item $O(n)$ is $n$

\item Prove $T(n)$ is $O(n)$ showing that $T(n) \leq c \cdot f(n)$ where $c > 0$ and $n_{0} \geq 0$ such that $ \forall  n \geq n_{0}$ \\ 

\centering $n+2\leq c \cdot n$  \\
\vspace{5pt}
when $n_{0} = 2$ and $c = 3$ \\
\vspace{5pt}
\centering $ 4 \leq 6$ is true\\
\vspace{5pt} 
since $c > 0$ and $ n_{0} \geq 0 $ where $ n \geq n_{0}$\\
\vspace{5pt}
 $T(n)$ is $O(n)$  $\blacksquare$

							
\end{itemize}
      \end{enumerate}

%--------------------------------------------------
% Problem 2
%--------------------------------------------------
\section*{Problem 2 (10 points each)}
Suppose an algorithm solves a problem of input size $n$ in at most the number
      of steps listed for each $T(n)$ given below. Calculate the Big-$\Theta$ (not
      just Big-O) for each $T(n)$. Show your work, including values for $c$ and
      $n_0$.
      \begin{enumerate}[label=(\arabic*)\setlength{\listparindent}{\parindent}]
 
\item 
$T(n)$ is $5$\\ 
\vspace{5pt} 
{$O(n)$ is $1$\\
\vspace{5pt} 
for $5 \leq c \cdot 1$\\
\vspace{5pt} 
 $n_{0} = 0$ and $c = 5$ gives  \\
\vspace{5pt} 
$5 \leq 5$ which is true \\
\vspace{5pt} 
so $T(n)$ is $O(n)$}\\
\begin{center}
\vspace{-130pt}
$T(n)$ is $5$\\ 
\vspace{5pt} 
$\Omega(n)$ is $1$\\
\vspace{5pt} 
for $5 \geq c \cdot 1$\\
\vspace{5pt} 
$n_{0} = 0$ and $c = 5$ gives  \\
\vspace{5pt} 
$5 \geq 5$ which is true \\
\vspace{5pt} 
so $T(n)$ is $\Omega(n)$
\end{center}
Since $T(5)$ is $O(1)$ and $T(5)$ is $\Omega(1)$ \\
\vspace{5pt}
$T(5)$ is $\Theta(1)$

\pagebreak

      \item 
$T(n)$ is $2n^2+1$\\
\vspace{5pt} 
$O(n)$ is $n^2$\\
\vspace{5pt} 
for $2n^2+1 \leq c \cdot n^2$\\
\vspace{5pt} 
 $n_{0} = 3$ and $c = 3$ gives  \\
\vspace{5pt} 
$19 \leq 27$ which is true \\
\vspace{5pt} 
so $T(n)$ is $O(n)$\\
\begin{center}
\vspace{-130pt}
$T(n)$ is $2n^2+1$\\ 
\vspace{5pt} 
$\Omega(n)$ is $n^2$\\
\vspace{5pt} 
for $2n^2+1 \geq c \cdot n^2$\\
\vspace{5pt} 
$n_{0} = 1$ and $c = 2$ gives  \\
\vspace{5pt} 
$3 \geq 2$ which is true \\
\vspace{5pt} 
so $T(n)$ is $\Omega(n)$
\end{center}
Since $T(2n^2+1)$ is $O(n^2)$ and $T(2n^2+1)$ is $\Omega(n^2)$ \\
\vspace{5pt}
$T(2n^2+1)$ is $\Theta(n^2)$


\noindent\makebox[\linewidth]{\rule{\paperwidth}{0.4pt}}

      \item $T(n)$ is $3n^4+2n^3+2n$\\
\vspace{5pt} 
$O(n)$ is $n^4$\\
\vspace{5pt} 
for $ 3n^4+2n^3+2n\leq c \cdot n^4$\\
\vspace{5pt} 
 $n_{0} = 1$ and $c = 8$ gives  \\
\vspace{5pt} 
$7 \leq 8$ which is true \\
\vspace{5pt} 
so $T(n)$ is $O(n)$\\
\begin{center}
\vspace{-135pt}
$T(n)$ is $3n^4+2n^3+2n$\\ 
\vspace{5pt} 
$\Omega(n)$ is $n^4$\\
\vspace{5pt} 
for $3n^4+2n^3+2n \geq c \cdot n^4$\\
\vspace{5pt} 
$n_{0} = 1$ and $c = 2$ gives  \\
\vspace{5pt} 
$7 \geq 2$ which is true \\
\vspace{5pt} 
so $T(n)$ is $\Omega(n)$\\
\end{center}

Since $T(3n^4+2n^3+2n)$ is $O(n^4)$ and $T(3n^4+2n^3+2n)$ is $\Omega(n^4)$ \\
\vspace{5pt}
$T(3n^4+2n^3+2n)$ is $\Theta(n^4)$

\noindent\makebox[\linewidth]{\rule{\paperwidth}{0.4pt}}

      \item $T(n)$ is $log(5\times 2^n)$\\
\vspace{5pt}
which simplifies to\\
\vspace{5pt}
 $log 5 + log 2^n$\\
\vspace{5pt}
$n + log 5$\\
\vspace{5pt}
 where the base of the log is 2 \\
\vspace{5pt} 
$O(n)$ is $n$\\
\vspace{5pt} 
for $ n + log 5\leq c \cdot n$\\
\vspace{5pt}
 $n_{0} = 2$ and $c = 3$ gives  \\
\vspace{5pt} 
$2+log 5 \approx 4.3219 \leq 6$ which is true \\
\vspace{5pt} 
so $T(n)$ is $O(n)$\\
\begin{center}
\vspace{-207pt}
$T(n)$ is $log(5\times 2^n)$\\
\vspace{5pt}
which simplifies to\\
\vspace{5pt}
 $log 5 + log 2^n$\\
\vspace{5pt}
$n + log 5$\\
\vspace{5pt}
 where the base of the log is 2 \\
\vspace{5pt} 
$\Omega(n)$ is $n$ \\
\vspace{5pt} 
for $n + log 5\geq c \cdot n$\\
\vspace{5pt}
$n_{0} = 1$ and $c = 2$ gives  \\
\vspace{5pt} 
$1+log5 \approx 3.3219 \geq 2$\\
\vspace{5pt} 
 which is true \\
\vspace{5pt} 
so $T(n)$ is $\Omega(n)$
\end{center}
Since $T(log(5\times 2^n))$ is $O(n)$ and $T(log(5\times 2^n))$ is $\Omega(n)$ \\
\vspace{5pt}
$T(log(5\times 2^n))$ is $\Theta(n)$

\pagebreak

      \item 
$T(n)$ is $4n^2 logn$\\
\vspace{5pt}
$O(n)$ is $n^2logn$\\
\vspace{5pt} 
for $ 4n^2 logn\leq c \cdot n^2logn$\\
\vspace{5pt}
 $n_{0} = 2$ and $c = 5$ gives  \\
\vspace{5pt} 
$16 \leq 20$ which is true \\
\vspace{5pt} 
so $T(n)$ is $O(n)$\\
\begin{center}
\vspace{-130pt}
$T(n)$ is $4n^2 logn$\\
\vspace{5pt}
$O(n)$ is $n^2logn$\\
\vspace{5pt} 
for $ 4n^2 logn\geq c \cdot n^2logn$\\
\vspace{5pt}
$n_{0} = 2$ and $c = 3$ gives  \\
\vspace{5pt} 
$16 \geq 12$ which is true \\
\vspace{5pt} 
so $T(n)$ is $\Omega(n)$
\end{center}
Since $T(4n^2 logn)$ is $O(n^2logn)$ and $T(4n^2 logn)$ is $\Omega(n^2logn)$ \\
\vspace{5pt}
$T(4n^2 logn)$ is $\Theta(n^2logn)$


      \end{enumerate}



\end{document}
